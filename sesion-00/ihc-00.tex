\documentclass[11pt]{beamer}
\usepackage{listings} % Include the listings-package
\usepackage[T1]{fontenc}
\usepackage[utf8]{inputenc}
\usepackage[english]{babel}
\usepackage{amsmath}
\usepackage{amssymb, amsfonts, latexsym, cancel}
\usepackage{float}
\usepackage{graphicx}
\usepackage{epstopdf}
\usepackage{subfigure}
\usepackage{hyperref}
%\usepackage{authblk}
\usepackage{blindtext}
\usepackage{booktabs} % Allows the use of \toprule, 
\usepackage{filecontents}
\usepackage{courier} %% Sets font for listing as Courier.
\usepackage{listings}
%\usepackage{listings, xcolor}
\lstset{
tabsize = 2, %% set tab space width
showstringspaces = false, %% prevent space marking in strings, string is defined as the text that is generally printed directly to the console
numbers = left, %% display line numbers on the left
commentstyle = \color{green}, %% set comment color
keywordstyle = \color{blue}, %% set keyword color
stringstyle = \color{red}, %% set string color
rulecolor = \color{black}, %% set frame color to avoid being affected by text color
basicstyle = \small \ttfamily , %% set listing font and size
breaklines = true, %% enable line breaking
numberstyle = \tiny,
}
\usepackage{caption}
\DeclareCaptionFont{white}{\color{white}}
\DeclareCaptionFormat{listing}{\colorbox{gray}{\parbox{\textwidth}{#1#2#3}}}
\captionsetup[lstlisting]{format=listing,labelfont=white,textfont=white}
\definecolor{urlColor}{rgb}{0.06, 0.3, 0.57}
\definecolor{linkColor}{rgb}{0.57, 0.0, 0.04}
\definecolor{fileColor}{rgb}{0.0, 0.26, 0.26}
\hypersetup{
    colorlinks=true,
    linkcolor=linkColor,
    filecolor=fileColor,      
    urlcolor=urlColor,
}
\urlstyle{same}
\setbeamercovered{transparent}
%\usetheme{Boadilla}
\usetheme{CambridgeUS}
%\usetheme{Berkeley}
%\usetheme{Warsaw}
%\usetheme{Madrid}

\title[Presentación]{\bf\Huge Debbie Stone Et Al}

\author[asucasairet@unsa.edu.pe, achavezn@unsa.edu.pe, agarciapu@unsa.edu.pe, gturpoto@unsa.edu.pe]
{
	Armando Braulio Chavez Nina \inst{1} \\
	Arnold Ismael Sucasaire Torres \inst{2} \\
	Ayrton Robins Garcia Puma \inst{3} \\
	Gustavo Jonathan Turpo Torres \inst{4} 
}
\institute[UNSA]
{
\inst{1}% 
System Engineering and Informatic Department\\
Production and Services Faculty\\
San Agustin National University of Arequipa
}

\date[2020-09-15]{\scriptsize{2020-09-15}}
%\logo{\includegraphics[width=3.0cm]{img/logo_unsa.jpg}}


\begin{document}

\begin{frame}
\titlepage
\end{frame}

\begin{frame}
\frametitle{Contenido}
\tableofcontents
\end{frame}

\section{Biografía}
\begin{frame}
\frametitle{Biografía}
\begin{itemize}
\item Debbie Stone es profesora de la facultad de matemáticas e informática de la OpenUniversity. Ha obtenido un B.A. (Hons) en psicología y una maestría en sistemas inteligentes. Su Ph.D. se completó en la Open University y se basó en estudios empíricos de diseñadores que realizaban actividades de diseño al principio del ciclo de vida del diseño de HCI. Estos estudios se realizaron con el fin de hacer recomendaciones sobre cómo se puede apoyar a los diseñadores en sus primeras tareas de diseño de HCI. Stone ha estado involucrado en la enseñanza de HCI a través del aprendizaje a distancia desde 1993, como autor de curso, tutor de curso y marcador de exámenes. Más recientemente, ha estado realizando trabajos de consultoría para la evaluación práctica de la usabilidad.
\end{itemize}
\end{frame}


\section{Libros}
\begin{frame}
\frametitle{Libros}
\item User Interface Design and Evaluation - Debbie Stone (2005)
{\includegraphics[width=5.0cm]{UIDE_2.png}}
{\includegraphics[width=5.0cm]{UIDE_1.png}}

\end{frame}

\section{Objetivos del Libro}
\begin{frame}
\frametitle{Objetivos del Libro}
\begin{itemize}
\item Proveer de información sobre la teoría de un buen diseño de Interfaz de Usuario {\bf (UI)} que mejore el rendimiento con la interacción del {\bf software} y los {\bf usuarios}.
\end{itemize}
\begin{itemize}
\item Enseñar a los diseñadores que atender a los usuarios y como se comportan en la práctica son puntos escenciales para poder desarrollar UI utilizable.
\end{itemize}
\begin{itemize}
\item Será capaz de utilizar distintas habilidades para así conocer a los usuarios y atender sus necesidades cuando deseen diseñar una UI también podrá evalua diferentes UI.
\end{itemize}
\end{frame}





\section{Referencias}
%References frame
\begin{frame}
\frametitle{Referencias}
\begin{itemize}
\item \href{https://studylib.net/doc/25182344/-debbie-stone--caroline-jarrett--mark-woodroffe--s-bookfi-}{User Interface Design and Evaluation}
\end{itemize}
\end{frame}

\end{document}
